
% Szkielet dla pracy inżynierskiej pisanej w języku polskim.

\documentclass[polish,bachelor,a4paper,oneside]{ppfcmthesis}


\usepackage[utf8]{inputenc}
\usepackage[OT4]{fontenc}


% Authors of the thesis here. Separate them with \and
\author{%
   Karol Lisiecki \album{106458} \and 
   Radosław Osten-Sacken \album{106520}}
\title{Robot autonomiczny....}                   % Note how we protect the final title phrase from breaking
\ppsupervisor{dr inż. Piotr Zielniewicz} % Your supervisor comes here.
\ppyear{2015}                                         % Year of final submission (not graduation!)


\begin{document}

% Front matter starts here
\frontmatter\pagestyle{empty}%
\maketitle\cleardoublepage%

% Blank info page for "karta dyplomowa"
\thispagestyle{empty}\vspace*{\fill}%
\begin{center}Tutaj przychodzi karta pracy dyplomowej;\\oryginał wstawiamy do wersji dla archiwum PP, w pozostałych kopiach wstawiamy ksero.\end{center}%
\vfill\cleardoublepage%

% Table of contents.
\pagenumbering{Roman}\pagestyle{ppfcmthesis}%
\tableofcontents* \cleardoublepage%

% Main content of your thesis starts here.
\mainmatter% 

\chapter*{TEMP Wstępny spis treści, zagadnienia}

\begin{enumerate}
	\item{Wstęp}
	\begin{enumerate}
		\item{Wstęp}
		\item{Motywacja, wstępne założenia}
		\item{Ewolucja celu pracy}
		\item{Wybór platformy (przegląd innych możliwości - Raspberry Pi..)}
		\item{Kolejność, podział pracy}
	\end{enumerate}
	
	\item{Platforma - software}
	\begin{enumerate}
		\item{Opis Androida, zalety}
		\item{Opis IOIO}
		\item{Przykładowe kody na IOIO}
	\end{enumerate}
	
	\item{Kod, inżynieria oprogramowania?}
	\begin{enumerate}
		\item{Środowisko pracy}
		\item{Podział kodu na paczki, interfejsy}
		\item{Git}
	\end{enumerate}
	
	\item{Platforma - hardware}
	\begin{enumerate}
		\item{Plexi}
		\item{Silniki i enkodery, arduino}
		\item{Czujniki (+ możliwe użycie wbudowanych sensorów telefonu)}
		\item{Zasilanie}
		\item{Doświetlenie}
		\item{Problemy}
		\item{PCB}
	\end{enumerate}
	
	\item{Analiza obrazu}
	\begin{enumerate}
		\item{Założenia, co chcemy znajdować}
		\item{Wybór OpenCV}
		\item{Potencjalnie rozważane metody (HoughTransform, Harris, metoda z uczeniem wzorca do XMLa)}
		\item{Problemy}
		\begin{enumerate}
			\item{Znajdowanie śmieci}
			\item{Podwójne obramowanie}
			\item{Słaba rozdzielczość}
		\end{enumerate}
		\item{Ostateczny algorytm}
	\end{enumerate}
	
	\item{Pattern}
	\begin{enumerate}
		\item{Porównywanie}
		\item{Akceptacja}
		\item{Przechowywanie}
		\item{Określanie pozycji na mapie}
	\end{enumerate}
	
	\item{Obstacle}
	\begin{enumerate}
		\item{Problem szumów}
		\item{Eliminacja szumów na kratownicy}
	\end{enumerate}

	\item{Algorytm jazdy}
	\begin{enumerate}
		\item{lol, nie ma}
	\end{enumerate}
	
	\item{Zdalny pulpit}
	
	\item{Podręcznik użytkownika}
	\begin{enumerate}
		\item{Wymagania co do telefonu}
		\item{Sposób obsługi aplikacji}
	\end{enumerate}
	
	\item{Zakończenie}
	\begin{enumerate}
		\item{Porównanie założeń i osiągniętego efektu}
		\item{Ocena telefonu z Androidem jako platformy przetwarzającej}
		\item{Możliwości dalszej rozbudowy projektu}
		\item{Możliwe zastosowania jako platforma edukacyjna}
	\end{enumerate}

\end{enumerate}
	  
\chapter{Wstęp}

lalas
\chapter{Platforma - software}

lalas
\chapter{Inżynieria oprogramowania}

lalas
\chapter{Hardware}

lalas
\chapter{Analiza obrazu}

lalas
\chapter{Pattern}

lalas
\chapter{Obstacle}

lalas
\chapter{Algorytm jazdy}

lalas
\chapter{Zdalny pulpit}

lalas
\chapter{Podręcznik użytkownika}

lalas
\chapter{Zakończenie}

lalas


% All appendices and extra material, if you have any.
\cleardoublepage\appendix%
%
\chapter{Parę słów o stylu \texttt{ppfcmthesis}}

\section{Różnice w stosunku do ,,oficjalnych'' zasad składu ze stron FCMu}

Autor niniejszego stylu nie zgadza się z niektórymi zasadami wprowadzonymi w oficjalnym 
dokumencie FCMu.\footnote{\url{http://www.fcm.put.poznan.pl/platon/dokumenty/dlaStudentow/egzaminDyplomowy/zasadyRedakcji}}
Poniższe elementy są składane nieco inaczej w stosunku do ,,oficjalnych'' wytycznych.

\begin{itemize}
    \item Promotor na stronie tytułowej jest umiejscowiony w centralnej osi pionowej strony (a
    nie po prawej stronie).
    
    \item Czcionka użyta do składu to nie Times New Roman.
    
    \item Spacje między tytułami akapitów oraz wcięcia zostały pozostawione takie, jak są zdefiniowane
    oryginalnie w pakiecie Memoir (oraz w \LaTeX{}u). Jeśli zdefiniowano ,,polską'' opcję składu,
    to będzie w użyciu wcięcie pierwszego akapitu po tytułach rozdziałów. Przy składzie ,,angielskim''
    tego wcięcia nie ma.

    \item Odwrócona jest kolejność rozdziałów \emph{Literatura} i \emph{Dodatki}.

    \item Na ostatniej stronie umieszczono stopkę informującą o prawach autorskich i programie
    użytym do składu.
    
    \item Nie do końca zgadzam się ze stwierdzeniem, iż ,,zamieszczanie list tabel, rysunków, 
    wykresów w pracy dyplomowej jest nieuzasadnione''. Niektóre typy publikacji zawierają tabele i rysunki, których
    skorowidz umożliwia łatwiejsze ich odszukanie. Ale niech będzie.

    \item Styl podpisów tabel jest taki sam, jak rysunków i odmienny od FCMowego. 
    Jeśli ktoś koniecznie chce mieć zgodne z wytycznymi
    podpisy, to zamiast \texttt{caption} niech użyje \texttt{fcmtcaption} do podpisywania tablic oraz
    \texttt{fcmfcaption} do podpisywania rysunków. Podpisy pod rysunkami pozostaną pełne, a nie skrócone (,,Rys.'').
    
    \item Styl formatowania literatury jest nieco inny niż proponowany przez FCM.
\end{itemize}



% Bibliography (books, articles) starts here.
\bibliographystyle{plalpha}{\raggedright\sloppy\small\bibliography{bibliography}}

% Colophon is a place where you should let others know about copyrights etc.
\ppcolophon{}

\end{document}
