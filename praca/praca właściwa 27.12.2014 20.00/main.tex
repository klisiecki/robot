
% Szkielet dla pracy inżynierskiej pisanej w języku polskim.

\documentclass[polish,bachelor,a4paper,oneside]{ppfcmthesis}


\usepackage[utf8]{inputenc}
\usepackage[OT4]{fontenc}


% Authors of the thesis here. Separate them with \and
\author{%
   Karol Lisiecki \album{106458} \and 
   Radosław Osten-Sacken \album{106520}}
\title{Robot autonomiczny....}                   % Note how we protect the final title phrase from breaking
\ppsupervisor{dr inż. Piotr Zielniewicz} % Your supervisor comes here.
\ppyear{2015}                                         % Year of final submission (not graduation!)


\begin{document}

% Front matter starts here
\frontmatter\pagestyle{empty}%
\maketitle\cleardoublepage%

% Blank info page for "karta dyplomowa"
\thispagestyle{empty}\vspace*{\fill}%
\begin{center}Tutaj przychodzi karta pracy dyplomowej;\\oryginał wstawiamy do wersji dla archiwum PP, w pozostałych kopiach wstawiamy ksero.\end{center}%
\vfill\cleardoublepage%

% Table of contents.
\pagenumbering{Roman}\pagestyle{ppfcmthesis}%
\tableofcontents* \cleardoublepage%

% Main content of your thesis starts here.
\mainmatter% 

\chapter*{TEMP Wstępny spis treści, zagadnienia}

\begin{enumerate}
	\item{Wstęp}
	\begin{enumerate}
		\item{Wstęp}
		\item{Motywacja, wstępne założenia}
		\item{Ewolucja celu pracy}
		\item{Wybór platformy (przegląd innych możliwości - Raspberry Pi..)}
		\item{Kolejność, podział pracy}
	\end{enumerate}
	
	\item{Platforma - software}
	\begin{enumerate}
		\item{Opis Androida, zalety}
		\item{Opis IOIO}
		\item{Przykładowe kody na IOIO}
	\end{enumerate}
	
	\item{Kod, inżynieria oprogramowania?}
	\begin{enumerate}
		\item{Środowisko pracy}
		\item{Podział kodu na paczki, interfejsy}
		\item{Git}
	\end{enumerate}
	
	\item{Platforma - hardware}
	\begin{enumerate}
		\item{Plexi}
		\item{Silniki i enkodery, arduino}
		\item{Czujniki (+ możliwe użycie wbudowanych sensorów telefonu)}
		\item{Zasilanie}
		\item{Doświetlenie}
		\item{Problemy}
		\item{PCB}
	\end{enumerate}
	
	\item{Analiza obrazu}
	\begin{enumerate}
		\item{Założenia, co chcemy znajdować}
		\item{Wybór OpenCV}
		\item{Potencjalnie rozważane metody (HoughTransform, Harris, metoda z uczeniem wzorca do XMLa)}
		\item{Problemy}
		\begin{enumerate}
			\item{Znajdowanie śmieci}
			\item{Podwójne obramowanie}
			\item{Słaba rozdzielczość}
		\end{enumerate}
		\item{Ostateczny algorytm}
	\end{enumerate}
	
	\item{Pattern}
	\begin{enumerate}
		\item{Porównywanie}
		\item{Akceptacja}
		\item{Przechowywanie}
		\item{Określanie pozycji na mapie}
	\end{enumerate}
	
	\item{Obstacle}
	\begin{enumerate}
		\item{Problem szumów}
		\item{Eliminacja szumów na kratownicy}
	\end{enumerate}

	\item{Algorytm jazdy}
	\begin{enumerate}
		\item{lol, nie ma}
	\end{enumerate}
	
	\item{Zdalny pulpit}
	
	\item{Podręcznik użytkownika}
	\begin{enumerate}
		\item{Wymagania co do telefonu}
		\item{Sposób obsługi aplikacji}
	\end{enumerate}
	
	\item{Zakończenie}
	\begin{enumerate}
		\item{Porównanie założeń i osiągniętego efektu}
		\item{Ocena telefonu z Androidem jako platformy przetwarzającej}
		\item{Możliwości dalszej rozbudowy projektu}
		\item{Możliwe zastosowania jako platforma edukacyjna}
	\end{enumerate}

\end{enumerate}
	  
\input{010-wstep.tex}
\input{020-platforma-software.tex}
\chapter{Inżynieria oprogramowania}

lalas
\input{040-hardware.tex}
\input{050-analiza-obrazu.tex}
\input{060-pattern.tex}
\input{070-obstacle.tex}
\input{080-algorytm-jazdy.tex}
\input{090-zdalny-pulpit.tex}
\input{100-manual.tex}
\input{110-zakonczenie.tex}


% All appendices and extra material, if you have any.
\cleardoublepage\appendix%
%\input{0a-zalacznik.tex}

% Bibliography (books, articles) starts here.
\bibliographystyle{plalpha}{\raggedright\sloppy\small\bibliography{bibliography}}

% Colophon is a place where you should let others know about copyrights etc.
\ppcolophon{}

\end{document}
